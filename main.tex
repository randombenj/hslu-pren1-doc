% Load the main ipa class
\documentclass[oneside]{modern}

% Define the author
\title{Linux on Z}
\author{Benjamin Fassbind und Felix Niederberger}
\date{\today}

\begin{document}
   
   % magic * chapter starts at 1 :) 
   \renewcommand{\thesection}{\arabic{section}}

  % Print whole bibliographie
  \nocite{*}

  \firstpage
    {Was unterscheidet Linux on z zu einem Linux auf x86}
    {Mainframe Topics}
    {\theauthor}

  \addtableofcontents

  \newpage
  
  \section*{Aufgabenstellung}
  \addcontentsline{toc}
    {section}
    {Aufgabenstellung}
  \subfile{aufgabenstellung}
  
  %\chapter{Kapitel 1}
  
  \section{Die x86 CPU Architektur}
  \subfile{x86-architecture}
  \section{Die Mainframe Architektur}
  \section{Linux}
  \subfile{linux}
  \section{Linux on IBM}
  \subsection{History}
  % https://www.radiotux.de/index.php?/archives/540-RadioTux-Sendung-Januar-2009.html
  \subsection{Products}
  \section{Unterschiede}
  \section{Fazit}

  \subfile{glossary.tex}

  % ------------------------------------------------------------------------------
  % Assemble the document with the multiple parts

  %\subfile{0-typographic-elements.tex}


  %\section{Glossar}
  %\chapter{Glossaries}

  \addglossary

  % Literaturverzeichnis
  \newpage
  \addcontentsline{toc}
    {section}
    {Literatur}

  \printbibliography[
    heading=subbibliography
  ]

  \newpage
  \listoffigures

\end{document}