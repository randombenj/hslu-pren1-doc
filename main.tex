\documentclass[oneside]{modern}

\title{Produktentwicklung 1}
\author{
\begin{tabular}{ l l }
  Burch Sven & <sven.burch@stud.hslu.ch> \\
  Fassbind Benjamin & <benjamin.fassbind@stud.hslu.ch> \\
  Jacob Yves & <yves.jacob@stud.hslu.ch> \\
  Meier Boas & <boas.meier@stud.hslu.ch> \\
  Paradiso Simone & <simone.paradiso@stud.hslu.ch> \\
  Roos Yannick & <yannick.roos@stud.hslu.ch>
\end{tabular}
}
\date{\today}

\begin{document}

   % magic * chapter starts at 1 :) 
   \renewcommand{\thesection}{\arabic{section}}
   % also break urls
   \makeatletter
   \g@addto@macro{\UrlBreaks}{\UrlOrds}
   \makeatother
   
   %\textcolor{light-gray}{\rule{\linewidth}{1pt}}

   %\PassOptionsToPackage{hyphens}{url}\usepackage{hyperref}

  % Print whole bibliographie
  \nocite{*}

  \firstpage
    {Autonomer Baugerüst-Roboter}
    {Hochschule Luzern, Gruppe 5}
    {\theauthor}

  \addtableofcontents
  
  \newpage
  \listoffigures
  
  \newpage
  \listoftables

  \newpage
  
  \subfile{glossary.tex}

  % ------------------------------------------------------------------------------
  % Assemble the document with the multiple parts

  \subfile{parts/0-doc-info}
  \subfile{parts/1-aufgabenstellung}
  \subfile{parts/2-vorgehen}
  \subfile{parts/3-technologierecherche}
  \subfile{parts/4-erste-evaluation}
  \subfile{parts/5-zweite-evaluation}
  \subfile{parts/6-teilkonzepte}
  \subfile{parts/7-funktionsmuster}

  \newpage
  \addglossary

  % Literaturverzeichnis
  \newpage
  \addcontentsline{toc}
    {section}
    {Literatur}

  \printbibliography[
    heading=subbibliography
  ]
  
  \begin{comment}\subfile{parts/Anhang}
  \end{comment}

\end{document}