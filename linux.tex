\subsection{Linux?}
Linux ist ein Betriebssystem.
Um genau zu sein ist Linux nur ein Betriebssystem Kernel.
Dieser ist unter der \acrshort{GNU} GPL v2 Lizenz veröffentlicht \cite{kernelfaq} und deshalb frei verfügbar.
Gestartet wurde das Projekt von Linus Torwalds der als Hobby einen freies Betriebssystem schreiben wollte.\cite{fossbytelinuxannouncement}
Als Benutzer kommt man dann mit einer Linux Distribution in Kontakt, 
wie beispielsweise \href{https://ubuntu.com/}{Ubuntu}, \href{https://www.debian.org/}{Debian} oder \href{https://www.redhat.com/en/technologies/linux-platforms/enterprise-linux}{Red Hat Enterprise Linux}. Es gibt allerdings sehr viel mehr als nur diese drei Distributionen. 
Eine umfassende Liste kann beispielsweise in der Wikipedia gefunden werden. \cite{linuxdistros}
Eine Distribution ist eine Software Zusammenstellung die aus dem 
Linux Kernel, dem \acrshort{GNU} Userland \cite{gnusoftware} und zusätzlicher Software besteht.

\subsection{Architektur}