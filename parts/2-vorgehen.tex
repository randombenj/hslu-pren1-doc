\section{Vorgehen}

Es wurde beschlossen, nach Scrum \cite{Wikipedia-Scrum} vorzugehen.
Scurm ist eine einfache und agile Methode zur Entwicklung von Projekten. Es wird häufig in der 
Software Entwicklung eingesetzt, wurde aber ursprünglich aus Studien von Manufakturen wie der Autoindustrie entwickelt \cite{Wikipedia-Scrum-History}.

Da es sich in \acrshort{pren}1 nicht um ein reines Software-, sondern um
ein Interdisziplinäres Projekt handelt, besteht die Gefahr, dass Aufgaben
mit niedrigerer Priorität nicht umgesetzt werden. Aus diesem Grund ist es
wichtig, bereits im Vorhinein mögliche Risiken zu identifizieren und deren Machbarkeit
abzuklären. Dies wurde mittels \nameref{sec:design-thinking} gemacht.

Um zu vermeiden dass niedriger priorisierte Aufgaben nicht vernachlässigt werden,
ist es wichtig ein gutes Backlogmanagement zu erstellen. Das heisst alle Aufgaben/Stories
zu definieren (siehe \nameref{sec:anforderungsliste}), zu schätzen und zu priorisieren.
Damit kann bereits früh festgestellt werden, welche Aufgaben zeitlich umgesetzt werden 
können.

\subsection{Organigram}

Scrum gibt die folgenden Rollen vor:

\begin{items}
  \item {\bf \acrfull{po}} \\
    Repräsentiert die Kunden des Produktes gegenüber dem Team 
    und ist verantwortlich um einen Business-Value zu generieren.
  \item {\bf \acrfull{sm}} \\
    Entfernt Hindernisse, die das Team an der Producktentwicklung hindern.
  \item {\bf Entwicklungsteam} \\
    Ist für die Entwicklung des Producktes zuständig
\end{items}

Die Rollen wurden gemäss dem Organigramm auf das Team verteilt. Natürlich sind 
im Rahmen der \acrshort{pren}1 Arbeit alle Teammitglider teil des Entwicklerteams.
Je nach Rolle kann aber mehr oder weniger Arbeit beispielsweise für das Backlogmanagement anfallen.

\image
  {img/organigram}
  {Organigram welches die Projektrollen festlegt}

\subsection{Design Thinking}
\label{sec:design-thinking}

Bei Design Thinking \cite{Wikipedia-Design-Thinking} geht es in erster Linie darum, strukturiert 
Ideen für die Umsetzung eines neuen Produktes zu finden. Auch soll das Design Thinking
dabei helfen, zu sehen, ob eine Idee umsetzbar ist. Dies wird dadurch erreicht,
dass sehr einfache (\acrshort{lofi}) Prototypen, beispielsweise aus Karton, 
gefertigt werden, um zu sehen, ob das gewünschte Konzept überhaupt umsetzbar ist.

Ein Ansatz des Design Thinking ist der Double Diamond Prozess. Auch dabei geht es
darum, dass man mit einem Problem konfrontiert wird, im hier dargelegten Fall durch die 
Aufgabenstellung, und versucht dabei die bestmögliche Lösung zu finden.

Die Idee dabei ist relativ einfach: Man unterteilt die Problemlösung in vier Teile.
Als erstes versucht man, möglichst breit gefächert das Problem zu verstehen. Dies geschieht
beispielsweise indem man sich mit den Kunden zusammensetzt und sich mit ihnen über das Problem unterhält.
Danach definiert man eine konkrete Problemstellung auf die man sich fokussiert.
In einer dritten Phase versucht man breit gefächerte Ideen zur Lösung des Problems zu finden.
Diese Ideen können auch durchaus unkonventionell sein, es sollte noch nichts ausgeschlossen werden.
Als nächstes definiert man aus den gefundenen Ideen
die besten, welche sich aufgrund von gewissen Kriterien für die Lösung eignen.
Diese klärt man mithilfe von Prototypen auf die Machbarkeit.

\image
  {img/double-diamond}
  {Design Thinking mithilfe von Double Diamond}

\section{Arbeitsjournal}

Dieses Kapitel beinhaltet eine Übersicht der Sprints und welche Tätigkeiten 
in diesen gemacht wurden.

\subsection*{Technologierecherche und Anforderungsliste}
\workday
    {1}
    {\ok Meilenstein 1 Erledigt. Alle Tasks abgeschlossen.}
    {
      Es wurde sich intensiv mit der Technologierecherche auseinandergesetzt.
      Dabei wurde versucht, wie im Design Thinking Prozess beschrieben, so breit wie möglich
      zu bleiben.
    }
    {
      Das Testat 1 konnte erfolgreich abgegeben werden mit verschiedenen Technologierecherchen in
      allen Bereichen.
    }
    \begin{figure}[H]
        \dayliboard{assets/sprint1.pdf}
        \caption{Scrum Board von Sprint 1}
    \end{figure}

\subsection*{Evaluation, Auswahl der optimalen Lösungskombination(en)}
\workday
    {2}
    {\ok Meilenstein 2 Erledigt. Bereits mit Meilenstein 3 angefangen.}
    {
      Im Meilenstein 2 wurden aus der Technologierecherche drei Lösungskonzepte erstellt
      und anschliessend evaluiert. Dabei wird der einfachste Ansatz weiterverfolgt. 
      Ganz nach dem Motto ``Keep it simple and stupid''.
    }
    {
      Aufgrund der zügigen Fortschritte bezüglich der ersten Evaluation, konnte rasch zur Evaluation der Teilkonzepte fortgeschritten werden.
    }
    \begin{figure}[H]
        \dayliboard{assets/trellosprint2.pdf}
        \caption{Scrum Board von Sprint 2}
    \end{figure}
    
\subsection*{Freigabe des Gesamtkonzeptes, Dokumentation zu 80\% erstellt}
\workday
    {3}
    {\ok Meilenstein 3 Erledigt.}
    {
      Die Dokumentation wurde soweit zu 80\% fertiggestellt. 
    }
    {
      Die Ideen konnten mithilfe der Funktionsmuster überprüft werden und geben einen verlässlichen Eindruck, wie der Roboter umgesetzt werden kann.
    }
    \begin{figure}[H]
        \dayliboard{assets/trellosprint2.pdf}
        \caption{Scrum Board von Sprint 3}
    \end{figure}