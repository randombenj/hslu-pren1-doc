\section{Vorgehen}

Wir haben uns als Team dazu entschlossen nach Scrum \cite{Wikipedia-Scrum} vorzugehen.
Scurm ist ein agiles Framework zur entwicklung von Projekten. Es wird häufig in der 
Software Entwicklung eingesetzt, wurde aber ursprünglich aus Studien von Manufaktur
Firmen wie der Autoindustrie entwickelt \cite{Wikipedia-Scrum-History}.

Da es sich in \acrshort{pren}1 nicht um ein reines Software-, sondern um
ein Interdisziplinäres Projekt handelt, besteht die Gefahr dass Features
mit niedrigerer Priorität nicht implementiert werden. Aus diesem Grund ist es
wichtig bereits im Vorhinein mögliche Risiken zu identifizieren und machbarkeiten
abzuklähren. Dies wurde mittels \nameref{sec:design-thinking} gemacht.

Um zu vermeiden dass niedriger Priorisierte Features nicht vernachlässigt werden
ist es wichtige ein gutes Backlogmanagement zu machen. Das heisst alle Features/Stories
zu definieren (siehe \nameref{sec:anforderungsliste}), zu schätzen und zu priorisieren.
Damit kann bereits früh festgestellt werden welche Features zeitlich implementiert werden 
können.

\subsection{Organigram}

Scrum gibt die folgenden Rollen vor:

\begin{items}
  \item {\bf \acrfull{po}} \\
    Repräsentiert die Kunden des Produktes gegenüber dem Team 
    und ist verantwortlich um einen Business-Value zu generieren.
  \item {\bf \acrfull{sm}} \\
    Entfernt Hindernisse, die das Team an der Producktentwicklung hindern.
  \item {\bf Entwicklungsteam} \\
    Ist für die Entwicklung des Producktes zuständig
\end{items}

Die Rollen wurden gemäss dem Organigram auf das Team verteilt. Natürlich sind 
sich im Rahmen der \acrshort{pren}1 Arbeit alle Teammitglider teil des Entwicklerteams.
Jeh nach Rolle kann aber mehr oder weniger Arbeit beispielsweise für Backlogmanagement anfallen.

\image
  {img/organigram}
  {Organigram welches die Projektrollen festlegt}

\subsection{Design Thinking}
\label{sec:design-thinking}

Bei Design Thinking \cite{Wikipedia-Design-Thinking} geht es in erster linie darum strukturiert 
Ideen für die Umsetzung eines neuen Produktes zu finden. Auch soll das Design Thinking
dabei helfen zu sehen ob eine Idee umsetzbar ist. Dies wird dadurch erreicht,
dass sehr einfache (\acrshort{lofi}) Prototypen, beispielsweise aus Karton, 
gefertigt werden um zu sehen ob das gewünschte Konzept überhaupt umsetzbar ist.

Ein ansatz des Design Thinking ist der Double Diamond Prozess. Auch dabei geht es
darum dass man mit einem Problem Konfrontiert ist, welches in unserem Fall durch die 
Aufgabenstellung, und versucht dabei die bestmögliche Lösung zu finden.

Die Idee dabei ist relativ einfach, man unterteilt die Problemlösung in vier teile.
Als erstes versucht man möglichst breitgefächert das Problem zu verstehen. Dies geschieht
beispielsweise indem man mit den Kunden zusammensitzt uns sich mit Ihnen über das Problem unterhält.
Danach definiert man eine Konkrete Problemstellung auf die man sich Fokusiert.
In einer dritten Phase versucht man breitgefächert Ideen zur Lösung des Problems zu finden.
Dies Ideen können durchaus auch unkonventionell sein, es sollte noch nichts ausgeschlossen werden.
Als nächstes definiert man aus den gefundenen Ideen
die besten, welche aufgrund von gewissen Kriterien sich für die Lösung eignen.
Diese beginnt man mithilfe von Prototypen auf die machbarkeit abzuklären und zu implementieren.

\image
  {img/double-diamond}
  {Design Thinking mithilfe von Double Diamond}