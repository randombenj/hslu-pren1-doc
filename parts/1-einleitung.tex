\section{Einleitung}

%Im Rahmen des Moduls Produktentwicklung 1 (Pren1) soll ein interdisziplinäres Team bestehend aus Informatik-, Maschinentechnik- und Elektrotechnikstudenten einen vollautonomen Roboter planen. Dieser Roboter soll in der Lage sein, die Teststrecke zweimal möglichst schnell und vor allem fehlerfrei zu absolvieren unter der Einhaltung aller vorgegebenen Regeln und Bedingungen.

%Diese Arbeit dient als Vorbereitung für das Modul Produktentwicklung 2 (Pren2), bei welchem diese Pläne in die Tat umgesetzt werden und das Team den Roboter baut, testet und am abschliessenden Wettbewerb der Hochschule Luzern teilnimmt.

%In der vorliegenden Arbeit wird die Lösungsfindung des Teams 5 erläutert. Die einzelnen Teilschritte vom Verarbeiten der Aufgabenstellung über die breite Ideensammlung und Recherche bis hin zur Validierung und Zusammenstellen des Lösungsansatzes werden auf den anschliessenden Seiten ausführlich gezeigt und erläutert. Die Prototypen und Modelle, welche für die Validierung der einzelnen Lösungsansätze gebaut wurden und den Grundstein für das Folgemodul bilden, werden ebenfalls bildlich dargestellt und erklärt.


In der vorliegenden Dokumentation wird im Rahmen des Moduls Produktentwicklung 1 (Pren1) ein autonomer Roboter geplant. Dieser Roboter muss ein Bild scannen und anschliessend eine Treppe mit Hindernissen völlig autonom erklimmen. Auf der oberen Plattform angekommen, muss er das Referenzbild finden und dies signalisieren.
Um einen autonomen Roboter zu bauen, braucht es viele verschiedenen Kompetenzen, weshalb die Arbeit wird in einem interdisziplinären Team bestehend aus Informatik-, Maschinentechnik- und Elektrotechnikstudierenden durchgeführt wird, welche ihr Wissen in das Projekt gewinnbringend einbinden können.

In diesem Bericht wird das Konzept erläutert, wie der Roboter diese Aufgabe zuverlässig erfüllen kann. Von anfänglichen Überlegungen zu einem sehr detaillierten Lösungsansatz. Dabei werden die gestellte Aufgabe in verschiedene Teilbereiche unterteilt. Um für jeden dieser Teilbereiche jeweils die geeignetste Lösung zu finden werden Funktionsmuster erstellt um zu unterscheiden, welcher Ansatz am zuverlässigsten funktioniert. Die Punkte, auf welche besonders geachtet werden sind dabei die Einfachheit und Zuverlässigkeit.
Die somit bestimmten Teillösungen werden anschliessend in den finalen Lösungsansatz zusammengefügt. 

Diese Arbeit dient als Vorbereitung für das Modul Produktentwicklung 2 (Pren2), bei welchem diese Pläne in die Tat umgesetzt werden und das Team den Roboter baut, testet und verbessert. Als Abschluss dieser Module findet am Ende ein Wettbewerb der Hochschule Luzern statt, an welchem all Teams teilnehmen.

