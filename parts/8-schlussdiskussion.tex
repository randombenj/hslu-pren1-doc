\newpage
\section{Schlussdiskussion}

\subsubsection{Entwicklungskosten}

Für die Funktionsmuster im PREN1 wurde nur ein Zahnriemen angeschafft. Alle anderen Komponenten waren bereits im Besitz der Teammitglieder oder konnten von der Hochschule ausgeliehen werden. In PREN2 werden dann die eigentlichen Kosten anfallen. Die nachfolgende Tabelle zeigt eine erste Übersicht über mögliche Ausgaben.

\begin{table}[htbp]
\centering
\begin{tabular}{ccc}
Bezeichnung & Anzahl &
Stückpreis(CHF) & Subtotal \\
Eins & Zwei & Drei \\
Vier & Fünf & Sechs \\
\end{tabular}
\caption[Tabelle]{Entwicklungskosten}
\label{tab:Kosten}
\end{table}

\subsubsection{Entwicklungsaufwand}

Der zeitliche Aufwand beinhaltet den Entwicklungsaufwand und die Erarbeitung der theoretischen Grundlagen. Dazu gehören die Funktionsmuster und die durchgeführten Tests. In allen Fachbereichen wurde viel Zeit in die Erarbeitung der Grundlagen investiert, um möglichst viele Lösungsansätze zu sammeln und diese auch zu evaluieren. Die Fachbereiche Maschinentechnik und Elektrotechnik haben Zeit investiert das Konzept für das Treppensteigen und die Fortbewegung aufzustellen. Die Prototypen für das Treppensteigen und die Fortbewegung wurden interdisziplinär gebaut und getestet. Der Fachbereich Informatik hat viel Aufwand für die Umgebungserkennung betrieben.

\begin{table}[htbp]
\centering
\begin{tabular}{ccc}
Was & Aufwand(h) &
Anzahl Personen & Anzahl Wochen & Subtotal(h) \\
Eins & Zwei & Drei \\
Vier & Fünf & Sechs \\
\end{tabular}
\caption[Tabelle]{Zeitlicher Aufwand}
\label{tab:Zeitaufwannd}
\end{table}


\subsubsection{Lessons Learned}



\subsubsection{Offene Punkte, Risiken, Ausblick}

