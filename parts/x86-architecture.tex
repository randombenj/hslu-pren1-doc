Die x86 Architektur basiert auf dem Intel 8086 Mikroprozessor. \cite{Wikipediax86}
Der 8086 war der 16-bit Nachfolger des 8-bit 8080 Mikroprozessors.
Der Name "x86" kommt vom 80{\it 86} Mikroprozessor und dessen Nachfolgern.

\image{img/8086}
    {Der 8086 Mikroprozessor von Intel}

Über die Jahre wurden zahlreiche Instruktionen hinzugefügt, die
Rückwärtskompatibilität wurde jedoch immer beibehalten, was den die x86
Systeme auch so weit verbreitet hat. Einmal für x86 kompiliert läuft ein
Binary auf jedem anderen x86 System. Unsere gesammte verteilung von
Software welche man vorkompiliert downloaden kann basiert auf dieser
Rückwärtskompatibilität. Aus diesem grund basieren auch die Mehrheit aller
Personal Computer und Server auf der x86 Architektur.



Das folgende Beispiel ist ein "Hello, World" programm in x86 Assembler
welches auf Linux den Syscall 1 (write) aufruft:

\begin{minted}{nasm}
; ----------------------------------------------------------------------------------------
; Writes "Hello, World" to the console using only system calls. Runs on 64-bit Linux only.
; To assemble and run:
;
;     nasm -felf64 hello.asm && ld hello.o && ./a.out
; ----------------------------------------------------------------------------------------

          global    _start

          section   .text
_start:   mov       rax, 1                  ; system call for write
          mov       rdi, 1                  ; file handle 1 is stdout
          mov       rsi, message            ; address of string to output
          mov       rdx, 13                 ; number of bytes
          syscall                           ; invoke operating system to do the write
          mov       rax, 60                 ; system call for exit
          xor       rdi, rdi                ; exit code 0
          syscall                           ; invoke operating system to exit

          section   .data
message:  db        "Hello, World", 10      ; note the newline at the end
\end{minted}


Eine Tabelle:

% Two tables because of page break
\begin{tabularx}{\textwidth}{ >{\hsize=.3\hsize}X >{\hsize=2.4\hsize}X >{\hsize=0.3\hsize}X }

    % Heading
    {\bf Nummer} & {\bf Beschreibung} & {\bf Priorität} \vspace{4mm} \\

    % Content
    A12 &
    Die produktive Version der Software wird auf einem offiziellen
    Roche Server betrieben. &
    \ok tief \\

    A13 &
    Die Softwarekomponenten sind so konfigurierbar, dass sie auch für
    einen offiziellen Debian Paket Server\footnote{
      z.B. ftp://ftp.debian.org
    } verwendet werden können. &
    \ok tief \\

    A14 &
    In der Suchmaske können reguläre Ausdrücke und Platzhalter
    verwendet werden. &
    \ok tief \\

    A15 &
    Ein auf der paketspezifischen Ansicht dargestelltes Paket kann
    online entpackt und inspiziert werden. &
    \ok tief \\

\end{tabularx}
