\section{Berechnungen}

\textbf{Abmasse}

Anhand der vorgegebenen Treppe wurden erste Massabschätzungen gemacht. Dabei werden die Tritthöhe und die Tritttiefe berücksichtigt.

\begin{figure}[H]
  \includegraphics[width=1
  \textwidth]{img/Treppensteigen/Erste Abmasse}
  \centering
  \caption{Skizze erste Abmasse}
\end{figure}

\textbf{Berechnung der Massen und ihre Schwerpunkte}\\

\textbf{Grundkörper}

Für die Masse und den Schwerpunkt des Grundkörpers wurde eine Berechnung mit den Hauptkomponenten gemacht. Dabei wurden Halterungen, Sensoren, Kameras und Normteile wie Schrauben oder Passfedern nicht berücksichtigt.

Masse Grundplatte: 0.15 kg

Masse Welle 1: 0.6 kg

Masse Welle 3: 0.3 kg

Masse Motor 1: 0.3 kg

Masse Motor 2: 0.3 kg

Masse Akku: 0.074 kg

Zusatzmasse: 0.95 kg\\

Gesamt Masse Grundkörper: 2.9 kg\\

Schwerpunkt Grundplatte: 100 mm

Schwerpunkt Welle 1: 50 mm

Schwerpunkt Welle 3: 79.4 mm

Schwerpunkt Motor 1: 79.4 mm

Schwerpunkt Motor 2: 18 mm

Schwerpunkt Akku: 180 mm

Schwerpunkt Zusatzmasse: 180 mm\\

Schwerpunkt Grundkörper: 105.2 mm

\begin{figure}[H]
  \includegraphics[width=0.8
  \textwidth]{img/Treppensteigen/SP Grundkörper final.png}
  \centering
  \caption{Grundkörper}
\end{figure}

\newpage

\textbf{Standfüsse}

Masse Standfuss: 0.51 kg

Schwerpunkt Standfuss: 184.8 mm

\begin{figure}[H]
  \includegraphics[width=0.8
  \textwidth]{img/Treppensteigen/SP Standfuss final.png}
  \centering
  \caption{Standfuss}
\end{figure}

\textbf{Verbindungsleisten}

Der Schwerpunkt der Verbindungsleisten wurde mittig gesetzt, um die Schwerpunktsberechnung zu vereinfachen.

Masse Verbindungsleiste: 0.19 kg

Schwerpunkt Verbindungsleiste: 140 mm

\begin{figure}[H]
  \includegraphics[width=0.8
  \textwidth]{img/Treppensteigen/SP Verbindungsleiste final.png}
  \centering
  \caption{Verbindungsleiste}
\end{figure}

\newpage

\textbf{Notwendige Momente bei den Hubbewegungen}

Die Momente, die gebraucht werden, konnten mit den ersten Abmassen, Massen und Schwerpunkten der Komponenten berechnet werden und dienen als Grundlage zur Auswahl der Motoren.

\textbf{Momente bei den Hubbewegungen:} (G:Grundkörper, S:Standfuss, L:Verbindungsleiste)

\textbf{1. Hubbewegung:}

Drehachse 1: Moment zum horizontalen Halten des Grundkörpers
\begin{align*}
    M_{Drehachse 1} &= F_{GG} * 0.0552\ m \\
    &= m_{G} * g * 0.0552\ m \\
    &= 2.9\ kg * 9.81\ m/s^2\ * 0.0552\ m \\
    &= \underline{\underline{1.6\ Nm}}
\end{align*}

Drehachse 2: grösstes Moment am Anfang, in Ausgangsstellung
\begin{align*}
    M_{Drehachse 2} &= F_{G2L} * 0.115\ m + F_{GG} * 0.1748\ m \\
    &= g * (m_{2L} * 0.115\ m + m_{G} * 0.1748)\ m \\
    &= 9.81\ m/s^2\ * (0.38\ kg * 0.115\ m + 2.9\ kg * 0.1748\ m) \\
    &= \underline{\underline{5.4\ Nm}}
\end{align*}

\textbf{2. Hubbewegung:}

Drehachse 1: grösstes Moment, wenn alle Teile horizontal
\begin{align*}
    M_{Drehachse 1} &= F_{G2S} * 0.2852\ m + F_{G2L} * 0.115\ m \\
    &= g * (m_{2S} * 0.2852\ m + m_{2L} * 0.115\ m) \\
    &= 9.81\ m/s^2 * (1.02\ kg * 0.2852\ m + 0.38\ kg * 0.115\ m) \\
    &= \underline{\underline{3.3\ Nm}}
\end{align*}

Drehachse 2: grösstes Moment, wenn Standfüsse horizontal
\begin{align*}
    M_{Drehachse 2} &= F_{G2S} * 0.0552\ m \\
    &= m_{2S} * g * 0.0552\ m \\
    &= 1.02\ kg * 9.81\ m/s^2 * 0.0552\ m \\
    &=\underline{\underline{0.6\ Nm}}
\end{align*}

Motor 1 liefert das Moment, dass in der 2. Drehachse wirkt und Motor 2 liefert das Moment, dass in der 1. Drehachse wirkt.\\
\\
M$_{erforderlich}$ Drehachse 1: 3.3 Nm\\

M$_{erforderlich}$ Drehachse 2: 5.4 Nm verteilt auf zwei Seiten

M$_{erforderlich}$ Drehachse 2 pro Seite: 2.7 Nm
