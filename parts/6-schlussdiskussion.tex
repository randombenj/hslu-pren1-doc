\newpage
\section{Schlussdiskussion}
Zum Schluss werden die Kosten für die Umsetzung des Konzeptes sowie der zeitliche Aufwand für die Entwicklung des Konzeptes aufgelistet. Ausserdem geben die Studierenden der Fachbereiche eine Auskunft über das Gelernte bei diesem Konzeptbericht. Abschliessend wird ein Ausblick für die nächsten Schritte im \acrshort{pren2} gegeben.
\subsection{Entwicklungskosten}
Für die Funktionsmuster im \acrshort{pren1} wurde nur ein Zahnriemen angeschafft. Alle anderen Komponenten waren bereits im Besitz der Teammitglieder oder konnten von der Hochschule ausgeliehen werden. Bei der Beschaffung der Hardware für das Modul \acrshort{pren2} werden die eigentlichen Kosten anfallen. Die nachfolgende Tabelle \ref{tab:kosten} zeigt eine erste Übersicht über mögliche Ausgaben. In dieser Übersicht sind die Stundenansätze für das Werkstattpersonal oder 3D-Drucker nicht berücksichtigt. Auch Komponenten, die aus der Mechanikwerkstatt oder dem Elektronik Labor der HSLU bezogen werden können, sind nicht inbegriffen.

\begin{center}
\centering
\begin{table}[H]
\begin{tabular}{|l|r|r|r|}
\hline 
\textbf{Bezeichnung} & \textbf{Anzahl} &
\textbf{Stückpreis [CHF]} & \textbf{Gesamtkosten [CHF]} \\
\hline 
Motor 1 & 1 & 39.00 & 39.00 \\
\hline
Motor 2 & 1 & 21.00 & 21.00 \\
\hline
H-Brücke T & 1 & 10.00 & 10.00\\
\hline
Zahnriemen für Prototyp & 1 & 5.00 & 5.00 \\
\hline
Mikro Switch & 2 & 2.60  & 5.20\\
\hline
Antriebssystem & 2 & 18.00  & 36.00\\
\hline
H-Brücke F & 1 & 3.00 & 3.00 \\
\hline
Raspberry Pi 3 & 1 & 40.00 & 40.00 \\
\hline
Pi-Kamera & 1 & 30.00 & 30.00 \\
\hline
\acrshort{tof} Sensor 1 & 2 & 12.30 & 24.60 \\
\hline
Ultraschallsensor & 2 & 2.50 & 5.00 \\
\hline
DCDC Converter & 1 & 40.00 & 40.00 \\
\hline
\acrshort{tof} Sensor 2 & 1 & 10.00 & 10.00 \\
\hline
Akkupack & 1 & 24.00  & 24.00 \\
\hline
Lautsprecher & 1 & 3.00 & 3.00\\
\hline 
Zahnriemen & 2 & 21.00 & 42.00\\
\hline
Zahnriemenräder & 4 & 12.00 & 48.00\\
\hline
div. Zahnräder & 8 & - & 110.00\\
\hline \hline 
 \textbf{Subtotal} &&& \textbf{CHF 495.80}\\
\hline 
\end{tabular}
\caption[Entwicklungskosten]{Entwicklungskosten}
\label{tab:kosten}
\end{table}
\end{center}

\newpage

\subsection{Entwicklungsaufwand}
Der zeitliche Aufwand, ersichtlich in der Tabelle \ref{tab:entwicklungsaufwand}, beinhaltet den Entwicklungsaufwand und die Erarbeitung der theoretischen Grundlagen. Dazu gehören die Funktionsmuster und die durchgeführten Tests. In allen Fachbereichen wurde viel Zeit in die Erarbeitung der Grundlagen investiert, um möglichst viele Lösungsansätze zu sammeln und diese auch zu evaluieren. Die Fachbereiche Maschinentechnik und Elektrotechnik haben Zeit investiert das Konzept für das Treppensteigen und die Fortbewegung aufzustellen. Die Prototypen für das Treppensteigen und die Fortbewegung wurden interdisziplinär gebaut und getestet. Der Fachbereich Informatik hat viel Aufwand für die Umgebungserkennung betrieben.

\begin{center}
\begin{table}[H]
\begin{tabular}{|l|r|r|r|r|}
\hline
\textbf {Was} & \textbf{Aufwand [h]} &
\textbf{Anzahl Personen} & \textbf{Anzahl Wochen} & \textbf{Subtotal [h]}\\
\hline
Pren Block Donnerstag & 3 & 6 & 14 & 252 \\
\hline
Pren Block Freitag & 3 & 6 & 14 & 252 \\
\hline
Freizeitaufwand (E) & 1 & 3 & 14 & 42 \\
\hline
Freizeitaufwand (I) & 1.5 & 2 & 14 & 42 \\
\hline
Freizeitaufwand (M) & 3 & 1 & 14 & 42 \\
\hline
Total: & & & & 630 \\ \hline
\end{tabular}
\caption[Entwicklungsaufwand]{Entwicklungsaufwand}
\label{tab:entwicklungsaufwand}
\end{table}
\end{center}

\subsection{Lessons Learned}
Eine interdisziplinäre Arbeit in dieser Zeit des Distance-Learning war für das Team eine Herausforderung. Dabei wurde ersichtlich, wie wichtig eine gute und effiziente Kommunikation ist. Mit den von der Hochschule zu Verfügung gestellten Programmen konnte die Kommunikation auch auf Distanz flexibel durchgeführt werden. Im Bereich des ganzen Teams war die Strukturierung des Projektes mittels Trello eine neue Erfahrung welche den Horizont erweiterte. Diese Art der Projektführung war motivierend. Weiter wurde dem Team bewusst, was Interdisziplinarität wirklich bedeutet. Nämlich, dass man die Aufgaben nicht einfach nach den Disziplinen aufteilen kann, sondern das man für die meisten Probleme das Wissen aus allen drei Disziplinen benötigt und eine intensive Zusammenarbeit gefragt ist. Somit konnten alle in die Spezialgebiete der Kollegen hinein blicken und dabei neue Erfahrungen sammeln. Beim schreiben dieses Dokumentes wurden viele Informationen ausgetauscht, damit die teilweise schwer verständliche Fachsprache auch interdisziplinär verständlich wird.

Nachfolgend gehen die einzelnen Studiengänge noch auf die spezifischen Lektionen ein, welche sie während diesem Modul gelernt haben.

Die ET-Studierenden haben sich ausgiebig mit Motoren und deren Ansteuerung beschäftigt. Ebenfalls wurden dutzende Datenblätter nach den richtigen Komponenten für die Motoren, Akkus und Sensoren durchforscht. In diesem Umfang war das definitiv eine neue Erfahrung.

Für die Informatiker, war die Arbeit mit OpenCV, TensorFlow und Keras in der Kombination mit Python etwas Neues. Der Lösungsfindungsprozess der Problemstellungen Hinderniss- und Bilderkennung führten in die Bildverarbeitung und in das Deep Learning mit neuronalen Netzen, was später teilweise wieder verworfen wurde.
Im Zusammenhang mit der Bildverarbeitung wurden verschiedene Algorithmen wie z.B. die Canny Edge detection, Thresholding oder GaussianBlurr kennengelernt. 
Im Bereich Deep Learning wurde gelernt, wie ein Modell erstellt, trainiert und damit Bilder klassifiziert wird. Hierbei wurde unter anderem festgestellt, dass die Aufbereitung guter Trainingsdaten eine der grössten Herausforderungen darstellt. 

Für den Maschinentechnik-Studierenden war es eine neue Erfahrung theoretisch erarbeitete Konzepte mittels Funktionsmuster auf ihre Machbarkeit zu prüfen. Eine Idee wurde ausgearbeitet, überprüft und anschliessend verbessert. Dieses Vorgehen ist sehr wichtig und konnte in diesem Projekt sehr gut angewandt werden.
Dadurch, dass er auf Mithilfe seiner Kollegen angewiesen war, erhielt er Einblick in die anderen Fachbereiche.






\subsection{Offene Punkte, Risiken, Ausblick}

Nach dem Abschluss dieses Moduls, geht es direkt weiter mit dem Folgemodul \acrshort{pren2}. Dabei gilt es, den hier geplanten Roboter zu bauen. Dabei wird versucht, so früh wie möglich ein erstes Grundgerüst zu bauen, damit die Elektrokomponente und die Software frühzeitig getestet werden können. Diejenigen, welche nicht mit dem Bau des Grundgerüsts beschäftigt sind, werden die jeweiligen Komponenten unter Einhaltung des Budgets bestätigen, bestellen und erste Kompatibilität machen.
Das Ziel des Team 5 ist es, einen funktionierenden Prototyp nach 7 Semesterwochen vorweisen zu können. Somit bleibt genügend Zeit für die Optimierung und Verfeinerung.

\newpage

\subsubsection{Ausblick Elektrotechnik}
In diesem Modul konnte bereits nachgewiesen werden, dass die geplanten Teile auch ausgelesen und verwendet werden können. Im \acrshort{pren2} gilt es, dass die finalen Elektrokomponenten ausgiebig in ihrer Kompatibilität getestet werden und die entsprechenden Steuerungscodes angepasst werden.

Einen offenen Punkt bilden die Sensoren, welche im fertigen Gerüst getestet werden müssen, um die richtigen Einstellungen vorzunehmen.

\subsubsection{Maschinentechnik}
Für \acrshort{pren2} muss schnellstmöglich das mechanische Grundgerüst detailliert und Pläne dazu erstellt werden, um dieses bauen zu können. Die mechanischen Einkaufsteile müssen bestellt werden und die Werkstücke müssen dem Werkstattpersonal in Auftrag gegeben, oder selber gefertigt werden. Das Grundgerüst ist die Basis der weiteren Arbeiten in den Bereichen Elektronik und Informatik. 


\subsubsection{Informatik}
Als Erstes für \acrshort{pren2} geplant ist, dass die Hindernisserkennung mit Deep Learning realisiert wird. Um diese testen zu können, wird einzig das Raspberry Pi und die Kamera benötigt. Parallel dazu kann ebenfalls an der Piktogrammerkennung weiter gearbeitet werden, da sich diese ebenfalls gut unabhängig als einzelne Komponente entwickeln lassen. Der Fokus liegt jedoch klar in der Hindernisserkennung, da dieser Bereich als risikobehafteter gesehen wird.



\subsubsection{Risiken}
Mithilfe eines konsequenten Risikomanagementes können viele Risiken drastisch minimiert werden. Die Schwere einiger technische Risiken kann jedoch erst im \acrshort{pren2} vollumfänglich realisiert und minimiert werden. So ist z.B. die Drehung in der Nähe eines Hindernisses ein solcher Punkt (R14), für den das Team verschiedene Lösungsansätze hat. Diese müssen unter realen Bedingungen getestet und evaluiert werden.

Im Abschnitt \ref{sec:risikomanagement} ist die gesamte Risikoliste, welche im Verlauf von \acrshort{pren1} iterativ/inkrementell erstellt wurde, zu finden. Nachfolgen werden jedoch noch die Top-Drei technischen Risiken in Hinsicht auf \acrshort{pren2} aufgelistet.

\begin{center}
\begin{table}[H]
    \begin{tabularx}{\textwidth}{|l|X|X|}
        \hline
        \textbf{ID} & \textbf{Titel} & \textbf{Vorbeugende Massnahmen} \\ \hline
        R15 & Anschlag bei Drehbewegung in der Nähe eines Hindernisses bei minimalem Durchgang von 400 mm. & Lösungskonzept erarbeiten oder Dimension so wählen, dass Rotation immer möglich ist. \\ \hline
        R18 & Hindernisse werden nicht erkannt & Früh im PREN2 testen und Lichtverhältnisse miteinkalkulieren. \\ \hline
        R17 & Nicht alle Aspekte bei der Berechnung der Hubbewegung miteinkalkuliert. Hubbewegung funktioniert nicht. & Möglichst früh in \acrshort{pren2}, die Hubbewegung testen.\\ \hline
    \end{tabularx}
    \caption{Top 3 Risiken in Hinsicht auf PREN2}
    \label{tab:risikomanagement-ausblick}
\end{table}
\end{center}

Ein weiteres nicht technisches Risiko ist der Wissensverlust, da ein Teammitglied im \acrshort{pren2} nicht mehr dabei sein wird. Um diese Gefahr zu minimieren werden die Erfahrungen und das Wissen stetig ausgetauscht. Mit diesem Risiko geht einher, dass sich, falls eine neue Person dem Projekt beitritt, das neue Teammitglied bestmöglich in das existierende Projekt einarbeiten und somit effizient mitwirken kann.












