\section{Erste Evaluation}

Für den ersten Evaluationsschritt werden Killerkriterien formuliert. Diese Killerkriterien sind aus der Anforderungsliste und der Aufgabestellung abgeleitet. Es soll eine Erstevaluation gemacht werden, um die vielen recherchierten Lösungsmöglichkeiten einzugrenzen. Die verbleibenden, möglichen Teillösungen werden anschliessend in einem morphologischen Kasten zusammengetragen. Mit dem morphologischen Kasten werden die Teillösungen zu mehreren Gesamtlösungen kombiniert. Diese Gesamtlösungen werden dann in einem zweiten Evaluationsschritt mit einer Nutzwertanalyse gefolgt von einer Sensitivitätsanalyse bewertet. Die Sensitivitätsanalyse dient dazu, mit variabler Gewichtung der Kriterien die Ergebnisse der Nutzwertanalyse zu prüfen.

\subsection{Mechanisch}
\textbf{Killerkriterien}
\begin{enumerate} 
	\item Gerät funktioniert bei jeder Witterung. Gerät darf nicht von Wind und Regen beeinflusst werden.
	\item Das ganze Gerät muss sich auf dem Zielpodest (Podest am oberen Ende der Treppe) befinden.
	\item Die Treppenstufen sind aus Holz oder Aluminium (nicht magnetisch).
\end{enumerate}

\textbf{Ausgeschiedene Teilfunktionen}

\begin{longtable}[h]{l|l|l}
%\toprule
\textbf{Fortbewegung} & \textbf{Treppensteigen} & \textbf{Lenkung} 
\tabularnewline
%\midrule
\endhead
Normale Räder & Spezielle Treppenräder & Lenkachse 
\tabularnewline
Omnidrive – Allseitenräder & Hebemechanismus: 3-Teilig & Knicklenkung 
\tabularnewline
Omnidrive – Mecanumräder & Hebemechanismus: Raufkappen & Panzerlenkung 
\tabularnewline
Omnidrive – Fahrdrehmodul & Hebemechanismus: Aufstapeln und Ausfahren & Roomba - Prinzip 
\tabularnewline
\cellcolor{red} Propeller – Rotoren & \cellcolor{red} Katapult & 
\tabularnewline
Beine & Sprungfeder &
\tabularnewline
 & \cellcolor{red} Bahn über Treppe ausfahren für kleines Fahrzeug & 
\tabularnewline
 & \cellcolor{red} Schlange mit Magneten & 
\end{longtable}

Teilfunktion Fortbewegung:
\begin{itemize}
    \item Propeller – Rotoren scheiden aufgrund des 1. Killerkriteriums aus. Ein Flug ist wegen der ungewissen Witterung problematisch. Sollte es am Wettkampftag regnen und winden, so wird es schwierig für eine Drohne auf dem Kurs zu bleiben und sich auf dem Zielpodest richtig, beim Zielpiktogramm zu platzieren. Aus diesem Grund wird diese Teillösung nicht weiter in Betracht gezogen.
\end{itemize}

Teilfunktion Treppensteigen:
\begin{itemize}
    \item Die Teillösung ein kleines Fahrzeug mit Hilfe eines Katapults über die Treppe, auf das Zielpodest zu schiessen kann wegen dem 2. Killerkriterium nicht weiterverfolgt werden. Das ganze Gerät muss sich auf dem Zielpodest befinden. Dies ist mit einem Katapult nicht möglich, da zwar das kleine Fahrzeug, dass katapultiert wird sich im Ziel befindet, jedoch das Katapult an Ort und Stelle im Startbereich bleibt.
    \item Auch die Idee eine Bahn von einem grossen Fahrzeug über die Treppe auszufahren, um mit einem kleinen Fahrzeug diese Bahn zu befahren scheitert am 2. Killerkriterium. Auch hier wäre zwar das kleine Fahrzeug auf dem Zielpodest, aber das grosse Bahnläger-Fahrzeug würde im Startbereich verbleiben.
    \item Die Teillösung einer Schlange mit Magneten an den Enden, um mit Hilfe dieser Magneten an den Vorderseiten der Treppenstufen die Treppe heraufzuklettern fällt weg, aufgrund des 3. Killerkriteriums. Da die Treppenstufen aus Holz oder Aluminium sein werden, wird der Einsatz von Magneten nicht möglich sein, da diese Materialien nicht magnetisch sind.
\end{itemize}

\textbf{Verbleibende Teillösungen für die zweite Evaluation}

\begin{longtable}[h]{l|l|l}
%\toprule
\textbf{Fortbewegung} & \textbf{Treppensteigen} & \textbf{Lenkung} 
\tabularnewline
%\midrule
\endhead
Normale Räder & Spezielle Treppenräder & Lenkachse 
\tabularnewline
Omnidrive – Allseitenräder & Hebemechanismus: 3-Teilig & Knicklenkung 
\tabularnewline
Omnidrive – Mecanumräder & Hebemechanismus: Raufkappen & Panzerlenkung 
\tabularnewline
Omnidrive – Fahrdrehmodul & Hebemechanismus: Aufstapeln und Ausfahren & Roomba - Prinzip 
\tabularnewline
Beine & Sprungfeder &
\end{longtable}

\subsection{Elektrotechnik}

\subsection{Auftragsquittierung}
\textbf{Killerkriterien}
\begin{enumerate} 
    \item Für den Bau der Teilfunktionsmuster in PREN 1 und für die Realisierung des Systems in PREN 2 stehen Ihnen als Team insgesamt CHF 500.- zur Verfügung. 
	\item Die Aufträge welche quittiert werden müssen, sind im vorhinein bekannt.
\end{enumerate}

\textbf{Ausgeschiedene Lösungsmöglichkeiten} 

OLED-Display:
\begin{itemize}
    \item Die Lösungsmöglichkeit, den Auftrag mittels eines OLED-Displays dem Publikum zu quittieren kann aufgrund des ersten Killerkriteriums nicht weiter verfolgt werden. Da die Kosten knapp sind, die Auftragsquittierung nicht zu den Knackpunkten dieses Projekts gehört und das OLED-Display verhältnismässig teuer ist, sollte man hier auf günstigere Möglichkeiten ausweichen.
\end{itemize}

Word Vorlesefunktion + Lautsprecher:
\begin{itemize}
    \item Die Lösungsmöglichkeit den Auftrag mittels der Word Vorlesefunktion auszugeben scheidet aufgrund des zweiten Killerkriteriums aus. Ein Vorteil dieser Variante ist, dass man grundsätzlich beliebige Sätze leicht erstellen und ausgeben kann. Da in diesem Projekt jedoch bekannt ist, was quittiert werden muss, fällt dieser Vorteil nicht mehr ins Gewicht.
\end{itemize}

\textbf{Verbleibende Lösungen für die zweite Evaluation}
\begin{itemize}
    \item LCD-Display
    \item Lautsprecher
    \item Audioplayer + Lautsprecher
    \item LED's
\end{itemize}

\subsection{Umgebungserkennung}
\textbf{Killerkriterien}
\begin{enumerate} 
    \item Das Gerät muss die Aufgabe autonom bewältigen. Es ist nicht erlaubt, das Gerät von einem externen stationären Rechner aus oder fremd zu steuern.
\end{enumerate}

\textbf{Ausgeschiedene Lösungsmöglichkeiten}

Google Vision API:
\begin{itemize}
    \item Die Google Vision API kann aufgrund des ersten Killerkriteriums nicht weiter vergolgt werden. Da es sich bei der Google Vision API um ein sozusagen fertiges Produkt handelt und man über ein REST API auf die Algorithmen für die Bilderkennung und das maschinelle Lernen zugreift.
\end{itemize}

CognitiveJ - Image Analysis mit Java:
\begin{itemize}
    \item Auch diese Lösung fällt aufgrund des ersten Killerkriteriums aus dem Rennen. Hier wird ebenfalls übers Internet auf vorgefertigte Lösungen zugegriffen, welche sich nicht auf dem Gerät selber befinden. Somit würde das Gerät indirekt über einem externen Server gesteuert werden.
\end{itemize}

\textbf{Verbleibende Lösungen für die zweite Evaluation}
\begin{itemize}
    \item TensorFlow 
    \item OpenCV
    \item PyTorch
    \item Scikit Learn
    \item Scikit Image
\end{itemize}