\section{Erste Evaluation}

Für den ersten Evaluationsschritt werden Killerkriterien formuliert. Diese Killerkriterien sind aus der Anforderungsliste und der Aufgabestellung abgeleitet. Es soll eine Erstevaluation gemacht werden, um die vielen recherchierten Lösungsmöglichkeiten einzugrenzen. Die verbleibenden, möglichen Teillösungen werden anschliessend in einem morphologischen Kasten zusammengetragen. Mit dem morphologischen Kasten werden die Teillösungen zu mehreren Gesamtlösungen kombiniert. Diese Gesamtlösungen werden dann in einem zweiten Evaluationsschritt mit einer Nutzwertanalyse bewertet.

\subsection{Mechanisch}
\textbf{Killerkriterien}
\begin{enumerate} 
	\item Gerät funktioniert bei jeder Witterung. Gerät darf nicht von Wind und Regen beeinflusst werden.
	\item Das ganze Gerät muss sich auf dem Zielpodest (Podest am oberen Ende der Treppe) befinden.
	\item Die Treppenstufen sind aus Holz oder Aluminium (nicht magnetisch).
	\item Die Hindernisse liegen zufällig flach oder hochkant verteilt auf der Treppe.
	\item Aufgrund des vierten Killerkriteriums muss auf einem Treppentritt nach rechts oder links ausgewichen werden können.
\end{enumerate}

\textbf{Ausgeschiedene Teilfunktionen}

\begin{longtable}[h]{l p{7cm} l}
%\toprule
\textbf{Fortbewegung} & \textbf{Treppensteigen} & \textbf{Lenkung} 
\tabularnewline
%\midrule
\endhead
Normale Räder & Spezielle Treppenräder & Lenkachse 
\tabularnewline
Omnidrive–Allseitenräder & Hebemechanismus: 3-Teilig & Knicklenkung 
\tabularnewline
Omnidrive–Mecanumräder & Hebemechanismus: Raufkappen & Panzerlenkung 
\tabularnewline
Omnidrive–Fahrdrehmodul & Hebemechanismus: Aufstapeln und Ausfahren & Roomba-Prinzip 
\tabularnewline
\cellcolor{red}Propeller,Rotoren & \cellcolor{red}Katapult & 
\tabularnewline
Beine & Sprungfeder &
\tabularnewline
 & \cellcolor{red}Bahn über Treppe ausfahren für kleines Fahrzeug & 
\tabularnewline
 & \cellcolor{red}Schlange mit Magneten & 
\end{longtable}
\captionlistentry[table]{Teilfunktionen Mechanik 1}

\begin{longtable}[h]{l p{7cm} l}
%\toprule
\textbf{Hindernisse überwinden} 
\tabularnewline
%\midrule
\endhead
Umgehen
\tabularnewline
\cellcolor{red}Überfahren
\tabularnewline
\cellcolor{red}Überfliegen
\tabularnewline
\cellcolor{red}Besteigen
\end{longtable}
\captionlistentry[table]{Teilfunktionen Mechanik 2}

Teilfunktion Fortbewegung:
\begin{itemize}
    \item Propeller und Rotoren scheiden aufgrund des ersten Killerkriteriums aus. Ein Flug ist wegen der ungewissen Witterung problematisch. Sollte es am Wettkampftag regnen und winden, so wird es schwierig für eine Drohne auf dem Kurs zu bleiben und sich auf dem Zielpodest richtig beim Zielpiktogramm zu platzieren. Aus diesem Grund wird diese Teillösung nicht weiter in Betracht gezogen.
 \end{itemize}

Teilfunktion Treppensteigen:
\begin{itemize}
    \item Die speziellen Treppenräder fallen wegen dem fünften Killerkriterium raus. Da den Hindernissen ausgeweicht werden muss, ist eine Verschiebung auf einem Treppentritt nötig. Dies ist mit einem Fahrzeug mit speziellen Treppenrädern nicht möglich.
    \item Die Teillösung ein kleines Fahrzeug mit Hilfe eines Katapults über die Treppe, auf das Zielpodest zu schiessen kann wegen dem zweiten Killerkriterium nicht weiterverfolgt werden. Das ganze Gerät muss sich auf dem Zielpodest befinden. Dies ist mit einem Katapult nicht möglich, da sich zwar das kleine Fahrzeug, dass katapultiert wird, im Ziel befindet, jedoch das Katapult an Ort und Stelle im Startbereich verbleibt.
    \item Auch die Idee eine Bahn von einem grossen Fahrzeug über die Treppe auszufahren, um mit einem kleinen Fahrzeug diese Bahn zu befahren scheitert am zweiten Killerkriterium. Auch hier wäre zwar das kleine Fahrzeug auf dem Zielpodest, aber das grosse Bahnleger-Fahrzeug würde im Startbereich verbleiben.
    \item Die Teillösung einer Schlange mit Magneten an den Enden, um mit Hilfe dieser Magneten an den Vorderseiten der Treppenstufen die Treppe heraufzuklettern fällt weg. Dies aufgrund des dritten Killerkriteriums. Da die Treppenstufen aus Holz oder Aluminium sein werden, wird der Einsatz von Magneten nicht möglich sein, da diese Materialien nicht magnetisch sind.
\end{itemize}

Teilfunktion Hindernisse überwinden:
\begin{itemize}
    \item Die Teillösungen die Hindernisse zu überfahren oder zu besteigen sind nicht möglich wegen dem vierten Killerkriteriums. Da die Hindernisse auch hochkant platziert werden können, können diese nicht einfach überfahren oder bestiegen werden.
    \item Die Teillösung die Hindernisse zu überfliegen fällt weg, weil die Rotoren und Propeller bereits ausgeschieden sind.
 \end{itemize}

\textbf{Verbleibende Teillösungen für die zweite Evaluation}

\begin{longtable}[h]{l p{7cm} l}
%\toprule
\textbf{Fortbewegung} & \textbf{Treppensteigen} & \textbf{Lenkung} 
\tabularnewline
%\midrule
\endhead
Normale Räder & Spezielle Treppenräder & Lenkachse 
\tabularnewline
Omnidrive – Allseitenräder & Hebemechanismus: 3-Teilig & Knicklenkung 
\tabularnewline
Omnidrive – Mecanumräder & Hebemechanismus: Raufkappen & Panzerlenkung 
\tabularnewline
Omnidrive – Fahrdrehmodul & Hebemechanismus: Aufstapeln und Ausfahren & Roomba - Prinzip 
\tabularnewline
Beine & Sprungfeder &
\end{longtable}
\captionlistentry[table]{Lösungen Mechanik 1}

\begin{longtable}[h]{l p{7cm} l}
%\toprule
\textbf{Hindernisse überwinden} 
\tabularnewline
%\midrule
\endhead
Umgehen
\end{longtable}
\captionlistentry[table]{Lösungen Mechanik 2}
\newpage


\subsection{Elektrotechnik}
\textbf{Killerkriterien}
\begin{enumerate} 
    \item Budget von CHF 500 nicht überschreiten
	\item Treppe und Treppenstufen finden/erkennen
	\item Hindernisse finden/erkennen
	\item Das Fahrzeug muss fähig sein zwei Läufe zu absolvieren.
	\item Das gesamte Gewicht von 3kg nicht überschreiten
	\item Das Gerät soll mit einem Not-Aus Taster sofort gestoppt werden können. 
	\item Autonomes Fahren unter allen zu erwartenden Licht- und Umgebungsbedingungen 
\end{enumerate}

\textbf{Ausgeschiedene Lösungsmöglichkeiten}

Orientierung
\begin{itemize}
    \item GPS hat eine Genauigkeit von ca. einem Meter und ist für eine präzise Positionierung auf der Treppe oder auf dem Startfeld ungeeignet.
\end{itemize}
Antrieb
\begin{itemize}
    \item Ein Verbrennungsmotor aufgrund des hohen Gewichts nicht geeignet. Ausserdem muss die Mechanische Energie zuerst in elektrische Energie für die elektrischen Bauteile umgewandelt werden.
    \item Ein Federwerk ist wegen der geringen Leistung nicht geeignet. Nur durch einmaliges Aufziehen des Federwerks, könnte der Roboter keine zwei Läufe absolvieren.
\end{itemize}
Not-Aus
\begin{itemize}
    \item Das Killerkriterium 6 verlangt einen Not-Aus Taster. Somit fallen die Funktionen 'Negative Beschleungiung' und 'Energie kappen' weg.
\end{itemize}
Energiequelle
\begin{itemize}
    \item Brennstoffzelle: Um eine genügend hohe Spannung und Kapazität für die elektrischen Bauteile, wie Prozessoren und Antriebsmotoren, zu erreichen, wäre eine hohe Anzahl an Zellen nötig. Dies würde das Gewicht des Roboters stark erhöhen. Mit weniger Zellen ist die Kapazität zu gering und es wäre gemäss Killerkriterium 4 nicht möglich zwei Läufe zu absolvieren.
    \item Photovoltaik: Gemäss dem Killerkriterium 7 muss der Roboter bei jeden Witterungsbedingnugen einsatzbereit sein. Durch bewölkten Himmel würde ein Photovolatikzelle nicht genügend Energie für den Betrieb des Roboters liefern.
\end{itemize}

\textbf{Verbleibende Lösungen für die zweite Evaluation}
\begin{center}
\begin{longtable}[h]{ l l l l}
 \textbf{Orientierung} & \textbf{Antrieb} & \textbf{Not-Aus} & \textbf{Energiequelle} \\ 
 \hline
 Tastsensor & Bürstenloser DC-Motor & Not-Aus-Schalter & Akku\\  
 Distanzsensor & Linearantrieb &  &\\
 Kamera & Schrittmotor & &\\
 TOF/Lidar & Servomotor & &\\
 Beschleunigungssensor & Getriebemotor & & 
\end{longtable}
\end{center}
\captionlistentry[table]{Lösungen Elektrotechnik}






\newpage
\subsection{Auftragsquittierung}
\textbf{Killerkriterien}
\begin{enumerate} 
    \item Für den Bau der Teilfunktionsmuster in PREN 1 und für die Realisierung des Systems in PREN 2 stehen Ihnen als Team insgesamt CHF 500.- zur Verfügung. 
	\item Die Aufträge welche quittiert werden müssen, sind im vorhinein bekannt.
\end{enumerate}

\textbf{Ausgeschiedene Lösungsmöglichkeiten} 

OLED-Display:
\begin{itemize}
    \item Die Lösungsmöglichkeit, den Auftrag mittels eines OLED-Displays dem Publikum zu quittieren kann aufgrund des ersten Killerkriteriums nicht weiter verfolgt werden. Da die Kosten knapp sind, die Auftragsquittierung nicht zu den Knackpunkten dieses Projekts gehört und das OLED-Display verhältnismässig teuer ist, sollte man hier auf günstigere Möglichkeiten ausweichen.
\end{itemize}

Word Vorlesefunktion + Lautsprecher:
\begin{itemize}
    \item Die Lösungsmöglichkeit den Auftrag mittels der Word Vorlesefunktion auszugeben scheidet aufgrund des zweiten Killerkriteriums aus. Ein Vorteil dieser Variante ist, dass man grundsätzlich beliebige Sätze leicht erstellen und ausgeben kann. Da in diesem Projekt jedoch bekannt ist, was quittiert werden muss, fällt dieser Vorteil nicht mehr ins Gewicht.
\end{itemize}

\textbf{Verbleibende Lösungen für die zweite Evaluation}
\begin{itemize}
    \item LCD-Display
    \item Lautsprecher
    \item Audioplayer + Lautsprecher
    \item LED's
\end{itemize}

\subsection{Umgebungserkennung}
\textbf{Killerkriterien}
\begin{enumerate} 
    \item Das Gerät muss die Aufgabe autonom bewältigen. Es ist nicht erlaubt, das Gerät von einem externen stationären Rechner aus oder fremd zu steuern.
\end{enumerate}

\textbf{Ausgeschiedene Lösungsmöglichkeiten}

Google Vision API:
\begin{itemize}
    \item Die Google Vision API kann aufgrund des ersten Killerkriteriums nicht weiter vergolgt werden. Da es sich bei der Google Vision API um ein sozusagen fertiges Produkt handelt und man über ein REST API auf die Algorithmen für die Bilderkennung und das maschinelle Lernen zugreift.
\end{itemize}

CognitiveJ - Image Analysis mit Java:
\begin{itemize}
    \item Auch diese Lösung fällt aufgrund des ersten Killerkriteriums aus dem Rennen. Hier wird ebenfalls übers Internet auf vorgefertigte Lösungen zugegriffen, welche sich nicht auf dem Gerät selber befinden. Somit würde das Gerät indirekt über einem externen Server gesteuert werden.
\end{itemize}

\textbf{Verbleibende Lösungen für die zweite Evaluation}
\begin{itemize}
    \item TensorFlow 
    \item OpenCV
    \item PyTorch
    \item Scikit Learn
    \item Scikit Image
\end{itemize}

\subsection{\acrshort{lofi} Prototypen}

Um ein erstes Gefühl dafür zu bekommen, ob eine Idee funktionieren kann, eignet es 
sich einen \acrfull{lofi} Prototypen zu bauen, um die Funktionsweise
des ``Frosch'' Prinzips für das Treppensteigen zu testen.

Als erstes wird geprüft, ob die Idee umsetzbar ist.
Eine Unsicherheit dabei ist der Schwerpunkt und ob das Modell kippt
bevor es die Treppenstufe erreicht. Ein erstes Kartonmodell zeigt,
dass die Treppen so bestiegen werden könnten.

In einer nächsten Evaluation soll das Prinzip in einem 1:1 Prototypen veranschaulicht und realisiert werden.

\image
 {img/frosch.png}
 {Treppensteigen mit dem "Frosch" prinzip}
