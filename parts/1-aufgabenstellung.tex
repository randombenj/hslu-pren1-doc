\section{Aufgabenstellung}

Das Ziel der \acrfull{pren} 1 und 2 ist das Planen und Verwirklichen eines vollkommen autonomen Roboters. Die Anforderungen an diesen Roboter werden anschliessend vereinfacht behandelt. 

\image
 {img/skizze-treppe}
 {Skizzierung der Treppe}
 
Das Team hat zwei Minuten, um den Roboter auf dem Spielfeld in dem 40x40cm grossen 
Startfeld zu positionieren (Blaue Zone) und letzte Vorkehrungen zu treffen. 
Danach wird der Roboter mittels eines einzelnen Tasters gestartet. 
Für alle folgenden Aufgaben hat er maximal vier Minuten zur Verfügung. 
Die autonome Maschine sucht zuerst im Startbereich ein Piktogramm 
(linkes grünes Feld), welches zufällig aus 5 möglichen Piktogrammen ausgewählt wurde. 

Hat der Roboter dieses Bild gefunden, bestätigt er dies dem Publikum. 
Nun muss die Trepper erklimmt werden. Um die Aufgabe zusätzlich zu erschweren,
werden auf der Treppe zufällig 2 verschiedene Ziegelsteinarten positioniert. 
Diese muss der Roboter selbstständig um- oder überfahren, ohne sie zu verschieben.
Gewährleistet wird dabei, dass die Steine nur auf der grossen Seite liegen 
und auf jeder Treppenstufe ein Durchweg von mindestens 40cm vorhanden ist. 
Ist die obere Plattform erreicht, steht der Roboter vor dem letzten Teil der Aufgabe. 
Aus allen 5 verfügbaren, dargelegten Piktogrammen muss die Kopie jenes aus dem 
Startbereich gefunden werden. Wurde das richtige identifiziert, 
muss dies wiederum signalisiert werden. Das Team darf,
wenn der Roboter einmal gestartet ist, 
keinen Einfluss mehr auf das Geschehen nehmen und hat maximal 2 solcher Durchläufe 
um die Aufgabe zu bestehen. Für diese Aufgaben darf die Maschine nicht mehr 
als 100cm vom Boden abheben oder das Geländer zur Hilfe nehmen. 
Eine Fortbewegung mittels Kettenantriebes ist ebenfalls nicht erlaubt. 

\subsection{Anforderungsliste}