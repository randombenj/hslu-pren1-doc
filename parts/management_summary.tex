\newpage
\section*{Management Summary}
Die vorliegende Projektarbeit des Moduls \acrfull{pren1} befasst sich mit der Planung eines Projektes, um praxisnahe Kompetenzen im eigenen Studienbereich, aber auch in der Zusammenarbeit mit anderen Studiengängen zu erlangen.

Die in dieser Arbeit verwendeten Vorgehen stammen aus den verschiedenen Studiengängen, was die Kompetenzen im Bereich Projektmanagement zusätzlich fördert.

Das gesetzte Ziel dieses Projekts ist die Planung eines vollautonomen Treppensteigroboters, welcher ein Gegenstand am unteren Ende einer Treppe erfasst und am oberen Ende wiedererkennt. Während des Treppensteigens müssen verschiedene Hindernisse automatisch über- oder umfahren werden. Die Steuerung des Roboters muss sich dabei im Roboter selbst befinden, Hilfestellung von externen Rechnern ist verboten. Das Ziel dieses Gerätes ist es, die vorgegebene Strecke in zwei Läufen ohne Probleme schnellstmöglich zu absolvieren.

Um das bestmögliche Ergebnis zu gewährleisten, geht die Gruppe 5 bestehend aus zwei Informatik-, einem Maschinentechnik- und drei Elektrotechnikstudenten auf möglichst viele Lösungsmöglichkeiten und Ansätze ein. Eine Breite Ideensammlung gewährleistet, dass keine potentiell gute Ansätze verworfen werden. Diese werden evaluiert und sortiert. Mit dem gewonnenen Wissen werden sinnvolle Lösungsvorschläge zusammengestellt, verfeinert und nach verschiedenen Kriterien wie Umsetzbarkeit und Entwicklungsaufwand bewertet. Aus dieser Evaluation ergibt sich ein Hauptlösungsansatz und ein Ansatz als Plan-B. Der vielversprechendste Lösungsansatz wird anschliessend verfeinert und weiter geplant.

Der gefundene Lösungsansatz \glqq Frosch\grqq{} hat sich während diesen Phasen bewährt und wird als Lösungsansatz eingereicht, da er simpel und vielversprechend ist. Ebenfalls ist es ein adaptives System, mit welchem gut auf Änderungen des Auftrags oder Probleme, aber auch allgemeine Verbesserungen schnell und flexibel reagiert werden kann. Die Treppe erklimmt der Roboter mit einem Klappmechanismus mit nur zwei Hubbewegungen und kann sich dank zweier unabhängig voneinander drehbaren Rädern schnell in alle Richtung bewegen. Eine Vielzahl an Sensoren überprüft dabei die Bewegungsabläufe und verhilft zu einem reibungsfreien Erfüllen der Aufgabe. Die Zusammensetzung und Funktionsweise der Komponenten wird in dieser Arbeit ausführlich fundiert dargestellt.
Im nächsten Schritt, während dem Modul Produktentwicklung 2, wird dieser Lösungsansatz gebaut, getestet und optimiert, um die Aufgabe möglichst zuverlässig erfüllen zu können.
