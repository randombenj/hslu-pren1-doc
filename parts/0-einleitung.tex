\section{Einleitung}

Im Rahmen des Moduls Produktentwicklung 1 (Pren1) soll ein interdisziplinäres Team bestehend aus Informatik-, Maschinentechnik- und Elektrotechnikstudenten einen vollautonomen Roboter planen. Das Ziel dieses Roboters ist es, die Teststrecke zweimal möglichst schnell und fehlerfrei zu absolvieren. Die Einhaltung aller vorgegebenen Regeln wird dabei vorausgesetzt.

Diese Arbeit dient als Vorbereitung für das Modul Produktentwicklung 2 (Pren2), bei welchem diese Pläne in die Tat umgesetzt werden und das Team den Roboter baut, testet und am abschliessenden Wettbewerb der Hochschule Luzern teilnimmt.
In der vorliegenden Arbeit wird die Lösungsfindung des Teams 5 erläutert. Die einzelnen Teilschritte vom Verarbeiten der Aufgabenstellung über die breite Ideensammlung und Recherche bis hin zur Validierung und Zusammenstellen des Lösungsansatzes werden auf den anschliessenden Seiten ausführlich gezeigt und erläutert. Die Prototypen und Modelle, welche für die Validierung der einzelnen Lösungsansätze gebaut wurden und den Grundstein für das Folgemodul bilden, werden ebenfalls bildlich dargestellt und erklärt.
